\documentclass[a4paper, 12pt]{article}
\usepackage[utf8]{inputenc}

\begin{document}

\title{Prosjektkravspesifikasjon for Genyx}
\author{Marius Geitle}
\date{\today}

\maketitle

\section{Introduksjon}
\subsection{Formål}
Formålet med dette dokumentet er å gi en detaljert beskrivelse av kravene til "Genyx". Det vil illustrere formålet og en komplett beskrivelse av utviklingen av systemet. Det vil også illustrere begrensningene, grensesnitt og interaksjon med ekstern programmvare. Dette dokumentet er ment for å bli presentert til Lars Emil Knudsen for godkjenning og som referanse for utvikling av første versjon.


\subsection{Omfang}
Genyx er en notat applikasjon som gjør det mulig for å benytte en android telefon for å gjøre notater under møter eller andre steder og effektivt kunne finne igjen gamle notater ved å bruke enten geografisk possisjon, tidsrom, fri-text søk, og/eller kategorisering.

Under notering vil brukeren kunne opprette et notat som består av flere sider, og sidene kan inneholde enten bilder som kan bli ikkedestruktivt annotert, tekst, eller håndtegnet grafikk slik som håndskrift eller tegninger.

Applikasjonen trenger tilgang til kaldenderprovider for å kunne knytte opp notater mot kalenderhendelser. Den trenger også tilgang til internett og både fin og grov posssisjon for geografisk markering av notater. Det vil også bli utredet om det er er hensiktsmessig å integrere med facebook for å kunne tagge notater med venner fra facebook.

\section{Overordnet beskrivelse}
\subsection{Produktperspektiv}


\subsection{produktfunksjoner}

\subsection{Brukertyper}

\subsection{Begrensninger}

\subsection{Antagelser og avhengigheter}

\section{Spesifikke krav}

\subsection{Krav til eksterne grensesnitt}

\subsubsection{Brukergrensesnitt}

\subsubsection{Maskinvaregrensesnitt}

\subsubsection{Programmvaregrensesnitt}

\subsubsection{Kommunikasjonsgrensesnitt}

\end{document}