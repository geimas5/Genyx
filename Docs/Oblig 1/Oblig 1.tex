\documentclass[a4paper, 12pt]{article}
\usepackage[utf8]{inputenc}
\usepackage{graphicx}


\begin{document}
\title{Android programmering - Oblig 1}
\date{\today}
\author{Marius Geitle}

\maketitle

\section{App beskrivelse}

Se eget dokument for app beskrivelsen.

\section{Teorispørsmål}
\subsection*{Forklar}
\subsubsection*{Intent}
En Intent benyttes for å beskrive en handling og kan bli sendt inn til foreksempel StartActivity for å starte en aktivitet.

\subsubsection*{View}
Et view er en basisen for alle byggeblokkene i android. For eksempel er TextView, ImageView osv. views som arver fra klassen View.

\subsubsection*{Activity}
Alle "skjermene" som brukeren ser er en activity, så når du foreksempel navigerer fra hovedsiden til en kontaktliste i applikasjonen så navigerer du fra en aktivitet til en annen.

\subsubsection*{Content provider}
Content providers er providere som gir tillgang til data på kryss av applikasjoner. De kan bli benyttet internt som en del av arketekturen i applikasjonen, men er primært ment for å gjøre data tilgjenglig på kryss av applikasjoner.

\subsubsection*{Service}
En service er en komponent som blir benyttet til å utføre lengre oppgaver. Men det er viktig å huske på at en service ikke jobber på en egen tråd, og er ikke en egen applikasjon.

\subsection*{Hva benyttes AndroidManifest.xml til?}
ActivityManifest.xml blir benyttet til spessifisere en del essensiell informasjon som app versjon, target og min versjoner, hvilke aktiviteter, intent-filter osv.

\subsection*{Forklar Activity Life Cycle?}
\includegraphics*[scale=0.5]{ActivityLifecycle.png}

\subsubsection*{onCreate()}
Denne metoden er kjørt når aktiviteten blir opprettet.

\subsubsection*{onStart()}
Denne metoden blir kjørt når aktiviteten blir synlig for brukeren.

\subsubsection*{onResume()}
Denne metoden blir kjørt når brukeren begynner å jobbe med aktiviteten.

\subsubsection*{onPause()}
Denne metoden blir kjørt når aktiviteten blir pauset, og en tidligere aktivitet blir gjennopptatt.

\subsubsection*{onStop()}
Denne metoden blir gjørt når aktiviteten ikke lengre er synlig for brukeren.

\subsubsection*{onDestroy()}
Denne metoden blir kjørt når aktiviteten blir fjernet fra systemet.


\subsubsection*{onRestart()}
Denne metoden blir kjørt når metoden blir startet fra stoppet tilstand.

\end{document}